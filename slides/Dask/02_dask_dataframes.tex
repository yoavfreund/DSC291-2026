\documentclass[aspectratio=169]{beamer}
\usetheme{Madrid}
\usecolortheme{default}

\usepackage{graphicx}
\usepackage{listings}
\usepackage{xcolor}
\usepackage{amsmath}
\usepackage{url}

% Code listing setup
\lstset{
    basicstyle=\ttfamily\footnotesize,
    keywordstyle=\color{blue}\bfseries,
    commentstyle=\color{green!60!black},
    stringstyle=\color{red},
    breaklines=true,
    frame=single,
    backgroundcolor=\color{gray!10},
    language=Python
}

\title{Dask DataFrames}
\subtitle{Parallel Pandas for Large Datasets}
\author{CSE255 - Scalable Data Analysis}
\date{\today}

\begin{document}

\frame{\titlepage}

\begin{frame}{DataFrame Architecture}
\begin{itemize}
    \item One Dask DataFrame = many pandas DataFrames
    \item Partitioned along index
    \item Each partition is a pandas DataFrame
    \item Operations applied to each partition in parallel
\end{itemize}

\vspace{0.5cm}
\textbf{Key Insight:}
\begin{itemize}
    \item Think of Dask DataFrame as a "meta-DataFrame"
    \item Contains references to many pandas DataFrames
    \item Each partition can be on different machines
\end{itemize}
\end{frame}

\begin{frame}[fragile]{Blocked Parallel Strategy}
\textbf{How It Works:}
\begin{enumerate}
    \item Data split into blocks (partitions)
    \item Operations applied to each block independently
    \item Results combined to form final answer
\end{enumerate}

\vspace{0.3cm}
\textbf{Example:}
\begin{itemize}
    \item 1TB dataset $\rightarrow$ 10 partitions of 100GB each
    \item Each partition processed in parallel
    \item Results merged at the end
\end{itemize}

\vspace{0.3cm}
\textbf{Visual Concept:}
\begin{verbatim}
Large Dataset
    |
[Partition 1] [Partition 2] [Partition 3] [Partition 4]
    |              |              |              |
[Process 1]   [Process 2]   [Process 3]   [Process 4]
    |              |              |              |
    +--------------+--------------+--------------+
                    |
              Combined Result
\end{verbatim}
\end{frame}

\begin{frame}[fragile]{Creating DataFrames: From Files}
\begin{lstlisting}
import dask.dataframe as dd

# Read CSV files
df = dd.read_csv('data/*.csv')

# Read Parquet files
df = dd.read_parquet('s3://bucket/data/*.parquet')
\end{lstlisting}

\textbf{Key Features:}
\begin{itemize}
    \item Supports glob patterns (\texttt{*}, \texttt{?}, \texttt{[...]})
    \item Automatic schema inference
    \item Lazy loading (no data read until \texttt{.compute()})
\end{itemize}

\vspace{0.3cm}
\textbf{What Happens:}
\begin{itemize}
    \item Dask scans file paths
    \item Infers schema from first file
    \item Creates task graph for reading
    \item Data not loaded until needed
\end{itemize}
\end{frame}

\begin{frame}[fragile]{Creating DataFrames: From Pandas}
\begin{lstlisting}
import pandas as pd
import dask.dataframe as dd

# Create pandas DataFrame
pandas_df = pd.DataFrame({
    'col1': range(1000),
    'col2': range(1000, 2000)
})

# Convert to Dask DataFrame
dask_df = dd.from_pandas(pandas_df, npartitions=4)
\end{lstlisting}

\textbf{Key Points:}
\begin{itemize}
    \item Convert existing pandas DataFrames
    \item Specify number of partitions
    \item Useful for testing or small-to-medium data
\end{itemize}

\vspace{0.3cm}
\textbf{Partitioning:}
\begin{itemize}
    \item \texttt{npartitions=4} splits data into 4 partitions
    \item Each partition is a pandas DataFrame
    \item Index determines how data is split
\end{itemize}
\end{frame}

\begin{frame}[fragile]{Lazy Evaluation in Action}
\textbf{What "Lazy" Means:}
\begin{itemize}
    \item \texttt{dd.read\_csv()} doesn't load data
    \item Just creates task graph
    \item \texttt{.compute()} triggers actual loading
\end{itemize}

\vspace{0.3cm}
\textbf{Example:}
\begin{lstlisting}
df = dd.read_csv('data/*.csv')  # Instant (no data loaded)
print(df)  # Shows metadata only
result = df.sum()  # Still no data loaded
final = result.compute()  # NOW data is loaded
\end{lstlisting}

\textbf{Visual Concept:}
\begin{itemize}
    \item Before compute: Lightweight graph structure
    \item After compute: Actual data in memory
    \item Scheduler optimizes execution order
\end{itemize}
\end{frame}

\begin{frame}[fragile]{Basic Operations: Inspection}
\begin{lstlisting}
df.head()      # Triggers computation (loads first partition)
df.tail()      # Triggers computation (loads last partition)
df.describe()  # Triggers computation (loads all partitions)
df.dtypes      # No computation needed (metadata)
df.columns     # No computation needed (metadata)
\end{lstlisting}

\textbf{Key Distinction:}
\begin{itemize}
    \item \textbf{Metadata operations}: No computation (fast)
    \item \textbf{Data operations}: Trigger computation (slower)
\end{itemize}

\vspace{0.3cm}
\textbf{When Computation Happens:}
\begin{itemize}
    \item \texttt{.head()}, \texttt{.tail()}: Loads partial data
    \item \texttt{.describe()}, \texttt{len()}: Loads all data
    \item \texttt{.dtypes}, \texttt{.columns}: No data loading
\end{itemize}
\end{frame}

\begin{frame}[fragile]{Filtering and Selection}
\begin{lstlisting}
# Filter rows
filtered = df[df['column'] > 100]

# Select columns
selected = df[['col1', 'col2']]

# Combined
result = df[df['value'] > 50][['col1', 'col2']]
\end{lstlisting}

\textbf{How It Works:}
\begin{itemize}
    \item Works exactly like pandas
    \item Applied to each partition independently
    \item Filtering reduces data size early
\end{itemize}

\vspace{0.3cm}
\textbf{Visual Concept:}
\begin{verbatim}
Partition 1: [row1, row2, row3, row4, row5]
    | filter(column > 100)
Partition 1: [row2, row4]  (filtered)
\end{verbatim}
\end{frame}

\begin{frame}[fragile]{GroupBy Operations}
\begin{lstlisting}
# Group by category
grouped = df.groupby('category')

# Aggregate
result = grouped['value'].sum()
result.compute()

# Multiple aggregations
result = grouped.agg({
    'value': ['sum', 'mean', 'count']
})
\end{lstlisting}

\textbf{Key Features:}
\begin{itemize}
    \item Distributed groupby
    \item Handles shuffles efficiently
    \item Works like pandas groupby
\end{itemize}

\vspace{0.3cm}
\textbf{Visual Concept:}
\begin{verbatim}
Partition 1: [A: 1, B: 2, A: 3]
Partition 2: [B: 4, A: 5, C: 6]
    | groupby
A: [1, 3, 5] -> sum = 9
B: [2, 4] -> sum = 6
C: [6] → sum = 6
\end{verbatim}
\end{frame}

\begin{frame}[fragile]{Aggregations}
\begin{lstlisting}
# Column-wise aggregations
df.sum()
df.mean()
df.count()
df.max()
df.min()

# Row-wise aggregations
df.sum(axis=1)
\end{lstlisting}

\textbf{How It Works:}
\begin{itemize}
    \item Applied per partition, then combined
    \item Efficient for large datasets
    \item Same API as pandas
\end{itemize}

\vspace{0.3cm}
\textbf{Example:}
\begin{verbatim}
Partition 1: sum = 100
Partition 2: sum = 200
Partition 3: sum = 150
    | combine
Total sum = 450
\end{verbatim}
\end{frame}

\begin{frame}[fragile]{Joins and Merges}
\begin{lstlisting}
# Merge two DataFrames
result = dd.merge(df1, df2, on='key')

# Left join
result = dd.merge(df1, df2, on='key', how='left')

# Multiple keys
result = dd.merge(df1, df2, on=['key1', 'key2'])
\end{lstlisting}

\textbf{Key Features:}
\begin{itemize}
    \item Handles large joins efficiently
    \item Partition-aware (optimizes data movement)
    \item Same API as pandas merge
\end{itemize}

\vspace{0.3cm}
\textbf{Performance:}
\begin{itemize}
    \item Dask optimizes join strategy
    \item Minimizes data shuffling
    \item Can handle joins larger than memory
\end{itemize}
\end{frame}

\begin{frame}[fragile]{Reading from S3}
\begin{lstlisting}
# Read Parquet from S3
df = dd.read_parquet('s3://bucket/data/*.parquet')

# Read CSV from S3
df = dd.read_csv('s3://bucket/data/*.csv')
\end{lstlisting}

\textbf{Key Features:}
\begin{itemize}
    \item Direct S3 access (no download needed)
    \item Uses IAM roles (no credentials needed on EC2)
    \item Works with partitioned data
\end{itemize}

\vspace{0.3cm}
\textbf{Setup Required:}
\begin{itemize}
    \item IAM role attached to EC2 instance
    \item Bucket permissions configured
    \item That's it! No credentials in code
\end{itemize}
\end{frame}

\begin{frame}[fragile]{Reading Parquet Files}
\textbf{Why Parquet?}
\begin{itemize}
    \item \textbf{Columnar format}: Read only needed columns
    \item \textbf{Efficient compression}: 3-10x smaller than CSV
    \item \textbf{Schema preservation}: Data types stored with data
    \item \textbf{Cross-platform}: Works with many tools
\end{itemize}

\vspace{0.5cm}
\textbf{Visual Concept:}
\begin{verbatim}
Parquet File:
+-- Column 1: [val1, val2, val3, ...]
+-- Column 2: [val1, val2, val3, ...]
+-- Column 3: [val1, val2, val3, ...]

Only read columns you need!
\end{verbatim}
\end{frame}

\begin{frame}[fragile]{Writing Results}
\begin{lstlisting}
# Write to Parquet
df.to_parquet('s3://bucket/output/')

# Write to CSV
df.to_csv('s3://bucket/output/*.csv')

# With options
df.to_parquet(
    's3://bucket/output/',
    compression='snappy',
    write_index=False
)
\end{lstlisting}

\textbf{Key Features:}
\begin{itemize}
    \item Writes partitioned files
    \item One file per partition
    \item Maintains partitioning structure
\end{itemize}

\vspace{0.3cm}
\textbf{Output Structure:}
\begin{verbatim}
output/
+-- part.0.parquet
+-- part.1.parquet
+-- part.2.parquet
+-- part.3.parquet
\end{verbatim}
\end{frame}

\begin{frame}[fragile]{Index Management}
\textbf{Index Determines Partitioning:}
\begin{itemize}
    \item Index values determine which partition data goes to
    \item Setting index can be expensive operation
\end{itemize}

\begin{lstlisting}
# Set index (may trigger shuffle)
df = df.set_index('date')

# Repartition
df = df.repartition(npartitions=8)

# Reset index
df = df.reset_index()
\end{lstlisting}

\textbf{Best Practice:}
\begin{itemize}
    \item Set index during read if possible
    \item Avoid setting index on large datasets unnecessarily
\end{itemize}
\end{frame}

\begin{frame}[fragile]{When Operations Trigger Computation}
\textbf{Explicit Triggers:}
\begin{itemize}
    \item \texttt{.compute()} - explicit trigger
    \item \texttt{len(df)} - triggers computation
    \item \texttt{df.head()} - triggers partial computation
\end{itemize}

\textbf{Lazy Operations:}
\begin{itemize}
    \item Most operations are lazy
    \item Build task graph only
    \item No data loaded until compute
\end{itemize}

\vspace{0.3cm}
\textbf{Example:}
\begin{lstlisting}
df = dd.read_csv('data/*.csv')  # Lazy
filtered = df[df['col'] > 100]  # Lazy
grouped = filtered.groupby('cat')  # Lazy
result = grouped.sum()  # Lazy
final = result.compute()  # TRIGGERS computation
\end{lstlisting}
\end{frame}

\begin{frame}[fragile]{Memory Management}
\textbf{Key Principles:}
\begin{itemize}
    \item Each partition fits in memory
    \item Operations stream through partitions
    \item \texttt{.persist()} caches in memory
\end{itemize}

\vspace{0.3cm}
\textbf{Example:}
\begin{lstlisting}
# Stream through partitions (low memory)
result = df.groupby('cat').sum().compute()

# Cache in memory (higher memory, faster)
df_persisted = df.persist()
result1 = df_persisted.groupby('cat').sum()
result2 = df_persisted.groupby('cat').mean()
\end{lstlisting}

\textbf{Memory Strategy:}
\begin{itemize}
    \item Default: Stream (low memory)
    \item Use persist: When reusing data (higher memory)
\end{itemize}
\end{frame}

\begin{frame}[fragile]{Performance Tips: Partition Size}
\textbf{Partition Size Guidelines:}
\begin{itemize}
    \item \textbf{Too small}: Overhead dominates (many small tasks)
    \item \textbf{Too large}: Memory issues (partition doesn't fit)
    \item \textbf{Rule of thumb}: 100MB-1GB per partition
\end{itemize}

\vspace{0.3cm}
\textbf{Adjusting Partitions:}
\begin{lstlisting}
# Repartition to different size
df = df.repartition(npartitions=8)  # Fewer, larger partitions
df = df.repartition(npartitions=32)  # More, smaller partitions
\end{lstlisting}

\textbf{Visual Concept:}
\begin{itemize}
    \item X-axis: Partition size
    \item Y-axis: Performance
    \item Sweet spot in middle (100MB-1GB)
\end{itemize}
\end{frame}

\begin{frame}[fragile]{Performance Tips: Operations}
\textbf{Optimization Strategies:}

\begin{enumerate}
    \item \textbf{Filter early}: Reduces data size
    \begin{lstlisting}
# Good: Filter first
df[df['value'] > 100].groupby('cat').sum()

# Bad: Groupby first
df.groupby('cat').sum()[result > 100]
    \end{lstlisting}

    \item \textbf{Use appropriate dtypes}: Smaller dtypes = less memory
    \begin{lstlisting}
df['id'] = df['id'].astype('int32')  # Instead of int64
    \end{lstlisting}

    \item \textbf{Avoid iterating over rows}: Use vectorized operations
\end{enumerate}
\end{frame}

\begin{frame}[fragile]{Common Pitfalls: Setting Index}
\begin{lstlisting}
# SLOW: Sets index after loading all data
df = dd.read_parquet('s3://bucket/data/*.parquet')
df = df.set_index('date')  # Expensive shuffle!

# BETTER: Set index during read
df = dd.read_parquet(
    's3://bucket/data/*.parquet',
    index='date'  # Index set during read
)
\end{lstlisting}

\textbf{Why It Matters:}
\begin{itemize}
    \item Setting index requires data shuffle
    \item Can be very expensive for large datasets
    \item Set index during read when possible
\end{itemize}
\end{frame}

\begin{frame}[fragile]{Common Pitfalls: Iterating}
\begin{lstlisting}
# BAD: Iterating over partitions
for partition in df.to_delayed():
    process(partition)

# BETTER: Use vectorized operations
df.apply(func, axis=1)

# BEST: Use built-in operations
df.groupby('cat').apply(func)
\end{lstlisting}

\textbf{Key Principle:}
\begin{itemize}
    \item Avoid Python loops
    \item Use vectorized operations
    \item Leverage Dask's parallelization
\end{itemize}

\vspace{0.3cm}
\textbf{Performance Impact:}
\begin{itemize}
    \item Loops: Sequential, slow
    \item Vectorized: Parallel, fast
\end{itemize}
\end{frame}

\begin{frame}[fragile]{Real-World Example: Flight Data}
\begin{lstlisting}
# Read flight data from S3
df = dd.read_parquet('s3://bucket/flights/*.parquet')

# Filter delayed flights
delayed_flights = df[df['dep_delay'] > 0]

# Group by airline
by_airline = delayed_flights.groupby('carrier').size()

# Compute result
result = by_airline.compute()

# Display
print(result)
\end{lstlisting}

\textbf{Complete Workflow:}
\begin{enumerate}
    \item Read data (lazy)
    \item Filter (lazy)
    \item Group and aggregate (lazy)
    \item Compute (executes everything)
\end{enumerate}

\vspace{0.3cm}
\textbf{What Happens:}
\begin{itemize}
    \item All operations build task graph
    \item Scheduler optimizes execution
    \item Data processed in parallel
    \item Results combined
\end{itemize}
\end{frame}

\begin{frame}{Summary}
\textbf{Key Takeaways:}
\begin{itemize}
    \item Dask DataFrame = many pandas DataFrames
    \item Lazy evaluation for efficiency
    \item Pandas-like API (familiar)
    \item Scales to large datasets
\end{itemize}

\vspace{0.5cm}
\textbf{Best Practices:}
\begin{itemize}
    \item Filter early
    \item Right-size partitions
    \item Use persist strategically
    \item Avoid unnecessary index operations
\end{itemize}

\vspace{0.3cm}
\textbf{Next Steps:}
\begin{itemize}
    \item Learn delayed and futures
    \item Explore distributed computing
    \item Practice with real datasets
\end{itemize}
\end{frame}

\end{document}

