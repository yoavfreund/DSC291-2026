\documentclass{beamer}
\usetheme{Madrid}

\title{Streaming Algorithms for Heavy Hitters:\\ Misra--Gries and Space-Saving}
\date{}

\begin{document}

\frame{\titlepage}

%-------------------------------------------------------
\section{Problem Setting}
%-------------------------------------------------------

\begin{frame}{Heavy Hitters in Data Streams}
We observe a stream of items:
\[
x_1, x_2, \ldots, x_m
\]

Goal: Find the most frequent items using very small memory.

\begin{itemize}
    \item Cannot store full frequency table.
    \item Want sublinear space, ideally $O(k)$.
    \item Focus on \textbf{heavy hitters}: items with frequency $> m/(k+1)$.
\end{itemize}
\end{frame}

%-------------------------------------------------------
\section{Misra--Gries Algorithm}
%-------------------------------------------------------

\begin{frame}{Misra--Gries Algorithm (1979)}
Maintain at most $k$ counters, each storing:
\[
(\text{item}, \text{count})
\]

For each incoming item $x$:
\begin{enumerate}
    \item If $x$ already tracked: increment its counter.
    \item Else if fewer than $k$ counters in use: insert $(x,1)$.
    \item Else: decrement all counters by 1 and delete those that hit 0.
\end{enumerate}

\bigskip
Deterministic and extremely simple.
\end{frame}

%-------------------------------------------------------

\begin{frame}{Example Illustration}
Suppose $k = 2$ and the stream is:
\[
a, b, c, a, c, c, b, c
\]

The algorithm maintains at most two counters.
\begin{itemize}
    \item When a new item arrives and both counters are occupied, decrement both.
    \item Items with low frequency are quickly eliminated.
\end{itemize}
\end{frame}

%-------------------------------------------------------
\section{Analysis of Misra--Gries}
%-------------------------------------------------------

\begin{frame}{MG: Key Properties}
Let $f(x)$ = true frequency of item $x$.

If an item appears more than $m/(k+1)$ times:
\[
f(x) > \frac{m}{k+1}
\quad\Rightarrow\quad
x \text{ will appear in the MG summary.}
\]

\begin{itemize}
    \item \textbf{No false negatives} above threshold.
    \item At most $k$ candidates stored.
\end{itemize}
\end{frame}

%-------------------------------------------------------

\begin{frame}{MG: Error Bound}
Each counter decrement corresponds to eliminating $(k+1)$ distinct items.

Total number of decrements:
\[
\leq \frac{m}{k+1}.
\]

Estimated count $c(x)$ satisfies:
\[
f(x) - c(x) \le \frac{m}{k+1}.
\]

Thus MG provides deterministic, additive error.
\end{frame}

%-------------------------------------------------------

\begin{frame}{MG: Space and Time}
\begin{itemize}
    \item Space: $O(k)$ counters.
    \item Worst-case time per update: $O(k)$
    \item Amortized time: $O(1)$
\end{itemize}

\medskip
MG is excellent for \textbf{approximate heavy hitters} with strong guarantees.
\end{frame}

%-------------------------------------------------------
\section{Space-Saving Algorithm}
%-------------------------------------------------------

\begin{frame}{Space-Saving (Metwally, Agrawal, Abbadi, 2005)}
Also maintains $k$ counters, but instead of decrementing all:

For arriving item $x$:
\begin{enumerate}
    \item If $x$ is tracked: increment its counter.
    \item Else:
        \begin{itemize}
            \item Find counter with minimum value $(y, c_{\min})$
            \item Replace $y$ with $x$
            \item Set new counter for $x$ to $c_{\min} + 1$
        \end{itemize}
\end{enumerate}

\bigskip
This makes heavy hitters dominate faster and improves accuracy.
\end{frame}

%-------------------------------------------------------

\begin{frame}{Why Space-Saving Works Well}
\begin{itemize}
    \item Low-frequency items are the ones being replaced.
    \item High-frequency items quickly accumulate large counts.
    \item Errors are much smaller than the MG bound in practice.
    \item Update cost is $O(1)$ using a heap or linked structure.
\end{itemize}
\end{frame}

%-------------------------------------------------------
\section{Comparison}
%-------------------------------------------------------

\begin{frame}{Comparison: Misra--Gries vs Space-Saving}
\begin{tabular}{l|c|c}
 & Misra--Gries & Space-Saving \\
\hline
Space & $O(k)$ & $O(k)$ \\
Updates & amortized $O(1)$ & $O(1)$ \\
Error & $\le m/(k+1)$ & typically much smaller \\
Deterministic & Yes & Yes \\
Behavior & decrement-all & replace-min \\
Best for & theory guarantees & practical top-$k$ accuracy
\end{tabular}
\end{frame}

%-------------------------------------------------------
\section{Conclusion}
%-------------------------------------------------------

\begin{frame}{Takeaways}
\begin{itemize}
    \item Misra--Gries: strong deterministic guarantees, additive error, simple.
    \item Space-Saving: improved empirical accuracy, optimized for top-$k$.
    \item Both scale to massive streams with $O(k)$ space.
\end{itemize}

Streaming heavy hitters is a solved problem—fast, accurate, and memory-light.
\end{frame}

\end{document}
